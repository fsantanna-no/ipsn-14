\documentclass{sig-alternate-ipsn13}
%\documentclass{llncs}

\usepackage{verbatim}
\usepackage{xspace}
\usepackage{caption}

\usepackage{color}
    \definecolor{light}{gray}{0.92}
    \definecolor{dark}{gray}{0.30}
    %\definecolor{light}{rgb}{.90,.90,.90}
    \definecolor{darkgreen}{rgb}{0,.50,0}
    \definecolor{darkblue}{rgb}{0,0,.50}
    \definecolor{darkred}{rgb}{.50,0,0}
    \definecolor{darkpur}{rgb}{.50,0,.50}

\usepackage{listings}
%\usepackage{textcomp}
\lstset{
%columns=fullflexible,
%basicstyle=\ttfamily,
escapeinside={||},
mathescape=true,
    language=C, % choose the language of the code
    basicstyle=\fontfamily{pcr}\selectfont\scriptsize\color{black},
    keywordstyle=\color{black}\bfseries, % style for keywords
    numbers=none, % where to put the line-numbers
    numberstyle=\tiny, % the size of the fonts that are used for the line-numbers
    backgroundcolor=\color{light},
    showspaces=false, % show spaces adding particular underscores
    showstringspaces=false, % underline spaces within strings
    showtabs=false, % show tabs within strings adding particular underscores
    %frame=single, % adds a frame around the code
    tabsize=2, % sets default tabsize to 2 spaces
    %rulesepcolor=\color{gray}
    captionpos=t, % sets the caption-position to bottom
    breaklines=false, % sets automatic line breaking
    %breakatwhitespace=false,
    numbersep=2em,
    emph={par,or,hor,do,end,loop,await,emit,input,every,event,call,with,command,%
          var,and,then,else,C,return,pure,deterministic,nohold,finalize,%
          each, abort, when, signal, FOREVER, class, spawn},
    emphstyle={\bfseries},
    commentstyle=\color{dark}\scriptsize,
    %xleftmargin=20pt,
    %xrightmargin=20pt,
    framesep=20pt,
    %upquote=true,
    %aboveskip={1.5\baselineskip},
}

\newcommand{\CEU}{\textsc{C\'{e}u}\xspace}
\newcommand{\code}[1] {{\small{\texttt{#1}}}}

\begin{document}

\title{Language Support for Dynamic Applications in Wireless Sensor Networks}

\begin{comment}
\author{
John Smith and Mary Smith}
\institute{University of North Texas\\
           Computer Science Department\\
           Denton, TX, 76203-1366\\
           john@unt.edu, mary@cs.unt.edu}
\end{comment}

\maketitle

\begin{abstract}

Due to safety and resource constraints, programming languages for Wireless 
Sensor Networks lack rich dynamic functionality such as heap-living objects and 
garbage collection.
%
Nonetheless, to cope with the concurrent nature of WSNs, network protocols 
commonly need to handle multiple incoming packets at the same time, thus 
requiring dynamic allocation to hold data buffers.
%
Furthermore, handling a single packet may involve multiple steps, such as 
reading from sensors and acknowledge transmissions, requiring additional 
dynamic memory to hold control-flow information.
%
Currently, both data and control memory are managed manually, an approach that 
is known to be hard and error prone.
%
In this work we show how to extend a synchronous concurrent language to provide 
safe and automatic management of dynamic memory.
Our strategy reconciles data and control state into the single concept of 
\emph{organisms}, which provide multiple lines of execution with an object-like 
interface.
%
Unlike objects in Java, organisms are lexically scoped, and their life cycle is 
controlled following the structure of the program, rather than by runtime 
reachability algorithms.
%
As a result, memory reclamation is efficient, providing rich dynamic 
functionality as in Java, with little overhead in comparison to C-based 
systems.

\begin{comment}
By default, references to organisms are not allowed to escape their scope, xxx 
safety.

can be optionally allocated on the stack, memory pool, or heap, depending on 
the level of flexibility required in the program.
In all cases, memory reclamation is made automatically following lexical scope, 
thus avoiding... and more efficiently than garbage collectors.

The data and control memory is then handled either by asynchronous 
communication primitives (in event-driven systems) or by predefined number of 
worker threads (in multi-threaded systems).

Dynamic behavior in applications in Wireless Sensor Networks is a reality.
Wireless Sensor Networks are
\end{comment}

\end{abstract}

\section{Introduction}

\begin{comment}
this distinction between data and control data shows that both memory and 
control need to be allocated dynamically.

allocate code dynamically

protothreads / ocram cannot dynamically allocate threads

they should use statically allocated with loops

probls: no composition (as they are global)
        no reuse of memory (because they are alway active)

Dynamic functionality in embedded systems is usually discouraged due to 
resource constraints.
However, some types of applications inherently require memory allocation.

As an example, protocols in sensor networks typically forward messages through nodes at a non-deterministic rate, given that the number of neighbors and transmission periods can vary.
Hence, many protocols require dynamic memory management to hold receiving messages until they are successfully forwarded.

A simple FIFO queue might not be always optimal because forwarding a message may involve multiple steps with delays (e.g. transmission acknowledgments).
In such scenario, the protocol would rather handle multiple messages at the same time, raising the possibility of a message received later be discarded first.
\end{comment}

Contributions:

\begin{itemize}
\item An abstraction mechanism that reconciles objects and threads in a single 
        concept.

\item Lexically scoped dynamic instances with efficient and automatic memory 
        reclamation.

\item Syntactic support for choosing between \emph{heap} and \emph{pool} 
        allocation, accounting respectively for more flexibility or 
efficiency/predictability.
\end{itemize}

%\section{State of the Art}

\begin{comment}
As far as we know, the only XXX programming language targeting WSNs that 
supports automatic memory management is Java~\cite{wsn.java} (through the 
\emph{Djarelling}~\cite{wsn.djar} and \emph{Takatuka} VMs).

all threads share a single stack~\cite{wsn.java}
quais os problemas disso?
\end{comment}

%\section{Memory Allocation Primitives}
\section{Background}

Regardless of whether memory is managed automatically or not, the underlying 
system ultimately requires basic allocation and reclamation primitives such as 
\emph{malloc} and \emph{free}.
%
Unfortunately, out-of-the-box \emph{malloc/free} is not suitable for embedded 
systems, which have different requirements in comparison to standard desktop 
systems.
%
For instance, many memory management hazards that are neglected in larger 
systems can, instead, be critical in embedded systems:

\begin{enumerate}
\item \textbf{Memory corruption:}
    Many embedded systems lack memory protection, and continuous allocations in 
    the heap may end up corrupting the stack (and vice versa).

\item \textbf{Run-time overhead:}
    Memory management requires extra run-time bookkeeping.
    Also, in the context of embedded systems, a predictable execution model can 
    be even more important than the fastest scheme on the average.

\item \textbf{Metadata overhead:}
    Metadata used by the memory manager can spend precious bytes (e.g. linked 
    lists of free blocks).

\item \textbf{Memory fragmentation:}
    For constrained memory platforms, unusable holes between and inside 
    allocated blocks (external and internal fragmentation, respectively) can 
waste a big percentage of available memory.

\item \textbf{Unreproducible execution:}
    Successive executions of the same program may allocate memory in different 
    ways, possibly leading to different outcomes (e.g. an allocation fail).

\item \textbf{Reclamation hazards:}
    Properly reclaiming memory is far from trivial.
    A missed reclamation leads to a memory leak that wastes memory, while 
    reclaiming a memory block still in use leads to a dangling pointer that 
will eventually crash the application.
\end{enumerate}

Note that reclamation hazards are inherent to systems that require an explicit 
free operation, because its correct use is left to programmers.
As an alternative, garbage collection eliminates reclamation hazards, but may 
incur unacceptable run-time overheads.

Note also that as the C standard is not specific about memory allocation 
behavior, \emph{malloc/free} can perform bad in all enumerated issues.
%
Hence, higher-level systems built on top of flawed primitives would inherit the 
same weaknesses.

A common alternative for embedded systems is to use \emph{memory pools} as the 
underlying scheme to manage dynamic allocation.
A memory pool is a static buffer of \emph{N} predefined fixed-sized blocks of 
memory that can be used by an application.
%
In the context of sensor networks, both TinyOS and Coniki operating systems 
offer and promote the use of memory pools%
\footnote{
TinyOS provides the \emph{Pool} component, while Contiki, the \emph{memb} 
library.
}
because most threats raised above can be alleviated:

\begin{enumerate}
\item
Given that the whole space for memory pools is static, dynamic allocations will 
never corrupt the stack.
For the same reason, the exact maximum stack limit can be known at compile 
time, helping to evaluate the risks of its growth during runtime.

\item
The run-time overhead is minimal because implementations typically use simple 
arrays to hold the memory blocks.
For instance, both allocation and reclamation are \emph{O(1)} in TinyOS.

\item
As associated metadata, TinyOS and Contiki pools use a single auxiliary vector 
of size \emph{N} to hold pointers for free blocks.

\item
Regardless of how allocation patterns in applications vary, memory pools will 
always guarantee a minimal \emph{N} of memory blocks.
Hence, external fragmentation is non-existent.
Internal fragmentation, however, can be an issue and is discussed further.

\item
Given that memory operations are simple and only handle fixed-size blocks, the 
execution is always deterministic and predictable.

\item
Memory pools are still manipulated through malloc/free-like operations.
Hence, all challenges to properly reclaim memory still hold.
\end{enumerate}

Internal fragmentation occurs when an allocated memory block is bigger than the 
requested size.
Given that memory pools can only handle fixed-size blocks, any allocation that 
requests smaller blocks will contain internal fragmentation (while allocation 
of bigger blocks always fail).
For applications that require blocks of different sizes, memory pools
This is the case for most Java applications, in which all non-primitive types 
are dynamically allocated.

TODO:
In our system, we promote the use of memory pools as the underlying allocation 
mechanism, but also support heap allocation if more flexibility is required.
%
To deal with memory reclamation, we support
- safe
- efficient
- high level

\section{The Programming Language C\'eu}

% TODO:
%- synchronous semantics
%- like Esterel, similar to Protothreads
%- differ from proto compositions, timers,events,locals,fins

\begin{figure}[t]
\begin{lstlisting}%[backgroundcolor=\color{white}]

// DECLARATIONS
input <type> <id>;      // external event
event <type> <id>;      // internal event
var   <type> <id>;      // variable

// EVENT HANDLING
await <id>;             // awaits event
emit  <id>;             // emits  event

// COMPOUND STATEMENTS
do <...> end            // scope block
<...> ; <...> ;         // sequence
if <...> then <...>     // conditional
         else <...> end
loop do <...> end       // repetition
    break               //   (escape loop)
finalize <...>          // finalization
    with <...> end

// PARALLEL COMPOSITIONS
par/and do <...>        // rejoins on both sides
      with <...> end
par/or do  <...>        // rejoins on any side
      with <...> end
par do <...>            // never rejoins
  with <...> end

// C INTEGRATION
_f();                   // C call (prefix `_')
native do <...> end     // block of native code
pure <id>;              // "pure" annotation
safe <id> with <id>;    // "safe" annotation
\end{lstlisting}
\rule{8.5cm}{0.37pt}
\caption{ Syntax of \CEU.\newline
{\small %\textmd{
}%}
\label{lst.syntax}
}
\end{figure}

high-level compositions
static safety guarantees

\begin{comment}
- deterministic and bounded execution
- shared memory concurrency
- integration with C
- locals and finalization
- first-class timers
- internal events
\end{comment}

\begin{figure}[t]
\begin{lstlisting}[numbers=left,xleftmargin=3em]
par do
    loop do
        await 250ms;
        _toggle(0);
    end
with
    loop do
        await 500ms;
        _toggle(1);
    end
with
    loop do
        await 1000ms;
        _toggle(2);
    end
end
\end{lstlisting}
\rule{8.6cm}{0.37pt}
\caption{
    An introductory example in \CEU that blinks three LEDs in parallel trails.
    The LEDs blink every $250ms$, $500ms$, and $1000ms$, respectively.
    \label{lst.blink}
}
\end{figure}

\CEU~\cite{ceu.sensys13} is a synchronous reactive language that supports 
multiple lines of execution---known as \emph{trails}---that continuously react 
to external input events broadcast by the environment.
%
The example in Figure~\ref{lst.blink} blinks three LEDs in parallel with 
different frequencies.
%
Symbols defined externally in $C$%
\footnote{Symbols for functions, types, globals, and constants available in the 
$C$ compiler for the target embedded platform.
}, such as \code{\_toggle} in the example, can be used seamlessly in \CEU by 
prefixing their names with an underscore.

By following the synchronous model, each trail in \CEU reacts to an occurring 
event with a \emph{run-to-completion} policy and in the order they appear in 
the code: the scheduler of \CEU is non-preemptive and deterministic.
%
Reactions to subsequent events never overlap, and the compiler ensures that 
programs do not contain unbounded loops, which could lead to unresponsiveness 
for incoming events.
%
The language runtime relies on a queue to hold incoming events while reactions 
execute to completion.
%
The disciplined execution model of \CEU simplifies the automatic analysis about 
concurrency aspects, enabling a number of static safety guarantees, such as 
race-free shared-memory and $C$ calls~\cite{ceu.sensys13}.
%
%Note in the example of Figure~\ref{lst.blink} that every second the three 
%trails awake at the same time, but the deterministic scheduler follows the 
%order they appear in the code (i.e., first \emph{LED 0} toggles, then 
%\emph{LED 1}, and then \emph{LED 2}).

Programs in \CEU are structured with standard imperative primitives, such as 
sequences, loops, and assignments.
The extra support for parallelism allows programs to wait for multiple events 
at the same time.
Trails await events without loosing context information, such as locals and the 
program counter, which eases reasoning in concurrent 
applications~\cite{sync_async.cooperative}.
%
The conjunction of parallelism with standard control-flow enable hierarchical 
compositions, in which self-contained blocks of code can be deployed 
independently.
%
To illustrate the power of compositions in \CEU, consider the two variations of 
the structure that follows:

\begin{minipage}[t]{0.35\linewidth}
\begin{lstlisting}
loop do
    par/and do
        <...>
    with
        await 1s;
    end
end
\end{lstlisting}
\end{minipage}
%
\hspace{1cm}
%
\begin{minipage}[t]{0.35\linewidth}
\begin{lstlisting}
loop do
    par/or do
        <...>
    with
        await 1s;
    end
end
\end{lstlisting}
\end{minipage}

In the \code{par/and} loop variation, the code in the first trail (represented 
as \code{...}) is repeated every second at minimum, as the second trail must 
also terminate to rejoin the \code{par/and} primitive and restart the loop.
%
In the \code{par/or} loop variation, if the code does not terminate within a 
second, the second trail rejoins the composition (cancelling the first trail) 
and restarts the loop.
%
The two structures represent, respectively, the \emph{sampling} and 
\emph{timeout} patterns, which are common in embedded applications.
%
Note that the body of \textbf{\code{<...>}} may contain arbitrary code with 
nested compositions and awaits, but the described patterns will work as 
expected.

\begin{figure}[t]
\begin{lstlisting}[numbers=left,xleftmargin=3em]
input void START, STOP;
input _PKT_t* RECEIVE, SENDDONE;
loop do
    await START;
    par/or do
        await STOP;
    with
        loop do
            var _PKT_t* pkt = await RECEIVE;
            finalize
                _PKT_send(_PKT_next(pkt), pkt);
            with
                _PKT_cancel(pkt);
            end
            await SENDDONE;
        end
    end
end
\end{lstlisting}
\rule{8.6cm}{0.37pt}
\caption{ A simple routing protocol in \CEU.%\newline
{\small %\textmd{
%TODO.
}%}
\label{lst.srp}
}
\end{figure}

As a more complete example, Figure~\ref{lst.srp} implements a simple routing 
protocol for WSNs.
%
The declared input events (lines 1-2) represent the external interface of the 
protocol that commands its execution.
The events \code{START}/\code{STOP} control if the protocol should be active or 
not.
The event \code{RECEIVE} notifies that a new packet should be routed through 
the protocol, while the event \code{SENDDONE} acknowledges a successful packet 
transmission (both events carry packet pointers).
%
The protocol enters the top-level loop and awaits the starting event (line
4);
upon request, the program spawns two other trails:
one to await the stopping event (line 6),
and another to actually route packets in a loop (lines 8-16).
%
Whenever a \code{STOP} occurs, the trail in line 6 awakes and terminates, also 
terminating the whole \code{par/or} which aborts the routing loop and restarts 
the top-level loop (waiting for the next request to start in line 4, again).
%
The routing loop successively awaits a new packet to arrive (line 9), requests 
to forward it (line 11), and awaits it to be transmitted (line 14).
%
Because the call to \code{\_PKT\_send} passes the local pointer \code{pkt} to 
the radio driver (which runs asynchronously with the protocol), the \CEU 
compiler obligates to include the \code{finalize} clause to safely release the 
pointer going out of scope (lines 10-14).
Now, suppose the application is awaiting \code{SENDDONE}, while the radio is 
transmitting the packet, and a \code{STOP} event occurs:
the program awakes in line 6 and terminates the \code{par/or}, also aborting 
the routing loop (lines 8-16).
In this case, the local pointer \code{pkt} goes out of scope, but the 
\code{finalize} properly requests the radio to cancel the packet transmission.

The implementation of Figure~\ref{lst.srp} relies on the high-level control 
mechanisms of \CEU, such as nested parallel compositions, await statements, and
finalization blocks, not requiring explicit global variables or top-level 
threads for control purposes.
%
The code as whole is self contained and has a clear interface with the external 
environment---it can be moved around or composable in bigger applications 
without a single modification.
%
However, the current code can forward a single packet at a time.
Note that while awaiting a \code{SENDDONE} acknowledge (line 15), the program 
is not awaiting \code{RECEIVE} (line 9) and, therefore, will miss all packets 
in the meantime.
%
In the lack of rich dynamic support, a possible solution is to add a queue or 
memory pool and manipulate the data buffers and transmission control state of 
each received packet explicitly.
For instance, this approach is taken in the reference implementation of the 
\emph{Source Routing Protocol} for TinyOS%
\footnote{TinyOS repository: 
\url{https://github.com/tinyos/tinyos-release/tree/tinyos-2_1_2/tos/lib/net/srp}}.
%
In Section~\ref{sec.TODO} we propose a solution that keeps the overall 
structure of Figure~\ref{lst.srp}, but, for every new message that arrives, it 
dynamically spawns a new instance of the sending code (lines 10-15) to run in 
parallel with the receiving loop.
This way, each message will be handled in separate, keeping the main receiving 
loop always ready for new requests.

\begin{comment}

The conjunction of parallel compositions with the synchronous execution model 
provides precise information about the control flow of applications to the 
compiler of \CEU, enabling a number of static guarantees with precise 
information about the control flow of applications
- safety: bounded mem/exec,
- other features

\begin{lstlisting}
 == TYPES:
    * PKT_t
        the packet data type for the protocol

 == FUNCTIONS:
    * PKT_send
        transmits a protocol packet
    * PKT_cancel
        cancels a packet transmition
    * PKT_next
        gets the next hop to forward

 == EVENTS:
    * START
        a request to start the protocol
    * STOP
        a request to stop the protocol
    * RECEIVE
        an arriving protocol packet
    * SENDDONE
        a packet transmition acknowledgement
\end{lstlisting}
\end{comment}

- assumption, next always succeed

\begin{itemize}
\item sequential reasoning
\item no state variables (as a consequence)
\item locals (as a consequence)
\item parallel compositions
\item finalize safe (and enforced) termination
\end{itemize}

TinyOS
- 4 callbacks

Protothreads
- state vars for start/stop

\subsection{A Simple Packet Forwarder}

Input events
C operations

\subsection{C\'eu Organisms}

\begin{figure}[h]
\begin{lstlisting}[numbers=left,xleftmargin=3em]
class Forwarder with
    var _PKT_t* pkt;
do
    var _PKT_t buf = *pkt;
    loop do
        finalize
            _PKT_send(_PKT_next(&buf), &buf);
        with
            _PKT_cancel(&buf);
        end
        var _PKT_t* sent = await SENDDONE;
        if sent == &buf then
            break;
        end
    end
end

loop do
    await START;
    par/or do
        await STOP;
    with
        do
            loop do
                var _PKT_t* pkt = await RECEIVE;
                spawn [10] Forwarder(pkt);
            end
        end
    end
end
\end{lstlisting}
\rule{8.6cm}{0.37pt}
\caption{ TODO.\newline
{\small %\textmd{
TODO.
}%}
\label{lst.TODO}
}
\end{figure}

\section{Dynamic Abstractions in C\'eu}

Applications frequently require multiple instances of an abstraction to coexist 
during runtime.
As an example, the routing protocol of Figure~\ref{lst.srp} requires to handle 
multiple data packets with the same behavior.
%
To avoid duplicating source code (and saving ROM), languages often provide 
abstraction and/or code reentrancy mechanisms: the same code is shared among 
instances, which only differ on their data and point of execution.
%
In traditional multi-threaded systems, code reentrancy is achieved with 
function declarations that are executed with different stacks and instruction 
pointers.
In object oriented languages, a \emph{class} encapsulates methods and 
properties, and can be instantiated as objects.

In \CEU, we designed an hybrid approach, which combines threads and classes in 
the so called \emph{organisms}.
An organism class is composed of an \emph{interface} and a single \emph{body}.
The interface exposes public variables and internal events that other organisms 
(and the top-level body) can refer.
The body of a class has access to all presented functionality provided by \CEU, 
such as parallel compositions, $C$ calls, timers, etc.
An organism is instantiated by declaring a variable of the desired class, and 
its body is automatically spawned in parallel with the enclosing block.
%A method can be simulated by exposing an internal event in the interface of 
%the class and using the same technique of Figure~\ref{lst.func}.
%
Just like a normal local variable, the life cycle of an organism is determined 
by the block it is declared.
Not only the memory and the visibility of an organism are associated to a 
scope, but also its executing body, i.e., when an organism goes out of scope, 
its memory is reclaimed and its body is aborted.
%
The allocation and reclamation of organisms are statically calculated, with no 
runtime overhead such as garbage collection.
\CEU organisms tie together automatic thread and memory management in the same 
and efficient mechanism.

\begin{figure}[t]
%\rule{8.5cm}{0.37pt}
\begin{minipage}[t]{0.45\linewidth}
\begin{lstlisting}
class Blink with
  var int led;
  var int dt;
do
  loop do
    await (dt) ms;
    _toggle(led);
  end
end

do
  var Blink l0(0,  250),
            l1(1,  500),
            l2(2, 1000);
  await 1min;
end
\end{lstlisting}
\end{minipage}
%
\hspace{0.0cm}
%
\begin{minipage}[t]{0.45\linewidth}
\begin{lstlisting}[numbers=left,xleftmargin=0em]
do
  var <Blink> l0, l1, l2;
  par/or do
    l0.led =    0;
    l0.dt  =  250;
    l1.led =    0;
    l1.dt  =  500;
    l2.led =    0;
    l3.dt  = 1000;
    await 1min;
  with
    loop do
      await (l0.dt) ms;
      _toggle(l0.led);
    end
    await FOREVER;
  with
    loop do
      await (l1.dt) ms;
      _toggle(l1.led);
    end
    await FOREVER;
  with
    loop do
      await (l2.dt) ms;
      _toggle(l2.led);
    end
    await FOREVER;
  end
end
\end{lstlisting}
\end{minipage}
\rule{8.6cm}{0.37pt}
\caption{ The blinking LEDs using organisms with equivalent expansion.\newline
{\small %\textmd{
We instantiate three \code{Blink} organisms to run in parallel with the scope 
they are declared.
}%}
\label{lst.blink.orgs}
}
\end{figure}

The first column of Figure~\ref{lst.blink.orgs} re-implements the blinking 
example of Figure~\ref{lst.blink} to run for 1 minute.
%
The \code{Blink} class exposes the \code{led} and \code{dt} variables to be 
configured by the application, which creates three instances with different 
initializations.
%
After the initializations, the organisms bodies run in parallel with the 
\code{do-end} block they are declared and their scope is limited by it (lines 
10-16).
%
After 1 minute the block will go out of scope, reclaiming all memory associated 
with the organisms and aborting their bodies.
%
The second column shows how the organisms bodies are expanded to run in a 
\code{par/or} together with the enclosing block they are declared.
The expansion is illustrative, i.e., the code is not duplicated.
Note that in the expansion, the bodies of the organisms are followed by 
\code{await~FOREVER}%
\footnote{\code{FOREVER} is a reserved keyword in \CEU, and represents an 
external input event that never occurs.}%
, meaning that only the enclosing block can terminate the \code{par/or}.

Figure~\ref{lst.srp} shows part of our port of the SRP routing 
protocol~\cite{wsn.teps} to \CEU.
The protocol specifies a fixed number of \emph{forwarders} responsible for 
routing received messages to neighbours based on a static table.
Given that a forwarder holds internal state (i.e. a message buffer and the 
forwarding activity), we define a \code{Forwarder} class and create multiple 
instances to serve requests.

The first column of Figure~\ref{lst.srp} shows the receiving loop of the 
protocol, which invokes \code{emit~go} when a message needs to be forwarded.
The event is declared as global, so that \code{Forwarder} instances have access 
to it.
The forwarders are declared in a vector, creating \code{COUNT} different 
instances.
As the vector is local, all instances are automatically killed when the 
protocol is stopped.
(Note the use of the start/stop pattern of Figure~\ref{lst.radio} again.)

The second column of Figure~\ref{lst.srp} shows the \code{Forwarder} class.
Initially, all forwarders are in the same state, waiting for the global event 
\code{go}.
Once the receiving loop emits the event in the top-level body, the forwarders 
awake in the order they were declared.
The first forwarder atomically sets the \code{gotcha} variable, indicating that 
the message will be handled and that other forwarders should ignore it:
all other forwarders will await again for the next \code{go} emission.
With this technique, we eliminated the need of an explicit queue.
In the case that all forwarders become busy, the \code{go} emission will be 
missed (with \code{gotcha=0}), acting just like a full queue.

%begin{comment}
Note that \CEU organisms are not global entities and do not use the heap for 
memory.
Instead, they are bounded to the scope they are declared, and all memory is 
statically allocated, just like \CEU does for standard local variables.
Also, when an organism goes out of scope, the same automatic bookkeeping of 
\code{par/or} compositions holds, all internal trails are killed and 
finalization blocks execute (if any).
Hence, the ``garbage collection'' for both the memory and code in organisms is 
efficient and static.
Although \CEU does not support dynamic creation (which could lead do unbounded 
memory), scoped organisms offer some degree of flexibility when compared to 
systems providing global objects only~\cite{wsn.virgil,wsn.flowtalk}.
%end{comment}

% TODO: ND access

\begin{figure}[t]
%\rule{8.5cm}{0.37pt}
\begin{minipage}[t]{0.45\linewidth}
{\small
\begin{verbatim}
event _fwd_t go;
loop do
  await SRP_START;
  par/or do
    await SRP_STOP;
  with
    var Forwarder[COUNT] fwds;
    <initialize fwds>
    loop do
      await SRP_RECEIVE;
      <receive or forward>
      if hops_left > 0 then
        <prepare fwd>
        emit go=&fwd;
      end
    end
  end
end
\end{verbatim}
}
\end{minipage}
%
\hspace{0.0cm}
%
\begin{minipage}[t]{0.45\linewidth}
%\fbox{
{\small
\begin{verbatim}
class Forwarder with
  <...>
do
  loop do
    fwd = await global:go;
    if fwd:gotcha then
      continue;
    end
    fwd:gotcha = 1;
    <send message>
    <...>
  end
end
\end{verbatim}
%}
}
\end{minipage}
\rule{8.5cm}{0.37pt}
\caption{ SRP forwarders as organisms.
\label{lst.srp.xxx}
}
\end{figure}

\section{Evaluation}

\section{Conclusion}

\section{12 pages max}

\bibliographystyle{splncs}
\bibliography{my,other}
\end{document}
