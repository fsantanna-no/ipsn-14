\documentclass{llncs}

\usepackage{verbatim}
\usepackage{xspace}
\usepackage{caption}

\usepackage{color}
    \definecolor{light}{gray}{0.87}
    \definecolor{dark}{gray}{0.30}
    %\definecolor{light}{rgb}{.90,.90,.90}
    \definecolor{darkgreen}{rgb}{0,.50,0}
    \definecolor{darkblue}{rgb}{0,0,.50}
    \definecolor{darkred}{rgb}{.50,0,0}
    \definecolor{darkpur}{rgb}{.50,0,.50}

\usepackage{listings}
%\usepackage{textcomp}
\lstset{
columns=fullflexible,
%basicstyle=\ttfamily,
escapeinside={||},
mathescape=true,
    language=C, % choose the language of the code
    basicstyle=\fontfamily{pcr}\selectfont\footnotesize\color{black},
    keywordstyle=\color{black}\bfseries, % style for keywords
    numbers=none, % where to put the line-numbers
    numberstyle=\tiny, % the size of the fonts that are used for the line-numbers
    backgroundcolor=\color{light},
    %showspaces=false, % show spaces adding particular underscores
    %showstringspaces=false, % underline spaces within strings
    showtabs=false, % show tabs within strings adding particular underscores
    %frame=single, % adds a frame around the code
    tabsize=2, % sets default tabsize to 2 spaces
    %rulesepcolor=\color{gray}
    captionpos=t, % sets the caption-position to bottom
    breaklines=false, % sets automatic line breaking
    %breakatwhitespace=false,
    numbersep=2em,
    emph={par,or,do,end,loop,await,emit,input,event,call,with,command,%
          var,and,then,else,C,return,pure,deterministic,nohold,finalize,%
          class,spawn,ms,s,min,every},
    emphstyle={\bfseries},
    commentstyle=\color{dark},
    %xleftmargin=20pt,
    %xrightmargin=20pt,
    framesep=20pt,
    %upquote=true,
    %aboveskip={1.5\baselineskip},
}

\newcommand{\CEU}{\textsc{C\'{e}u}\xspace}
\newcommand{\code}[1] {{\small{\texttt{#1}}}}

\begin{document}

\title{Language Support for Dynamic Applications in Wireless Sensor Networks}

\begin{comment}
\author{
John Smith and Mary Smith}
\institute{University of North Texas\\
           Computer Science Department\\
           Denton, TX, 76203-1366\\
           john@unt.edu, mary@cs.unt.edu}
\end{comment}

\maketitle

\begin{abstract}

Due to safety and resource constraints, programming languages for Wireless 
Sensor Networks lack rich dynamic functionality such as heap-living objects and 
garbage collection.
%
Nonetheless, to cope with the concurrent nature of WSNs, network protocols 
commonly need to handle multiple incoming packets at the same time, thus 
requiring dynamic allocation to hold data buffers.
%
Furthermore, handling a single packet may involve multiple steps, such as 
reading from sensors and acknowledge transmissions, requiring additional 
dynamic memory to hold control-flow information.
%
Currently, both data and control memory are managed manually, an approach that 
is known to be hard and error prone.
%
In this work we show how to extend a synchronous concurrent language to provide 
safe and automatic management of dynamic memory.
Our strategy reconciles data and control state into the single concept of 
\emph{organisms}, which provide an object-like interface with multiple lines of 
execution.
%
Unlike objects in Java, organisms are lexically scoped, and their life cycle is 
controlled following the structure of the program, rather than by runtime 
reachability algorithms.
%
As a result, memory reclamation is efficient, providing rich dynamic 
functionality as in Java, with little overhead in comparison to C-based 
systems.

\begin{comment}
By default, references to organisms are not allowed to escape their scope, xxx 
safety.

can be optionally allocated on the stack, memory pool, or heap, depending on 
the level of flexibility required in the program.
In all cases, memory reclamation is made automatically following lexical scope, 
thus avoiding... and more efficiently than garbage collectors.

The data and control memory is then handled either by asynchronous 
communication primitives (in event-driven systems) or by predefined number of 
worker threads (in multi-threaded systems).

Dynamic behavior in applications in Wireless Sensor Networks is a reality.
Wireless Sensor Networks are
\end{comment}

\end{abstract}

\section{Introduction}

\begin{comment}
this distinction between data and control data shows that both memory and 
control need to be allocated dynamically.

allocate code dynamically

protothreads / ocram cannot dynamically allocate threads

they should use statically allocated with loops

probls: no composition (as they are global)
        no reuse of memory (because they are alway active)

Dynamic functionality in embedded systems is usually discouraged due to 
resource constraints.
However, some types of applications inherently require memory allocation.

As an example, protocols in sensor networks typically forward messages through nodes at a non-deterministic rate, given that the number of neighbors and transmission periods can vary.
Hence, many protocols require dynamic memory management to hold receiving messages until they are successfully forwarded.

A simple FIFO queue might not be always optimal because forwarding a message may involve multiple steps with delays (e.g. transmission acknowledgments).
In such scenario, the protocol would rather handle multiple messages at the same time, raising the possibility of a message received later be discarded first.
\end{comment}

Contributions:

\begin{itemize}
\item An abstraction mechanism that reconciles objects and threads in a single 
        concept.

\item Lexically scoped dynamic instances with efficient and automatic memory 
        reclamation.

\item Syntactic support for choosing between \emph{heap} and \emph{pool} 
        allocation, accounting respectively for more flexibility or 
efficiency/predictability.
\end{itemize}

\section{State of the Art}

\begin{comment}
As far as we know, the only XXX programming language targeting WSNs that 
supports automatic memory management is Java~\cite{wsn.java} (through the 
\emph{Djarelling}~\cite{wsn.djar} and \emph{Takatuka} VMs).

all threads share a single stack~\cite{wsn.java}
quais os problemas disso?
\end{comment}

\section{Memory Allocation Primitives}

Regardless of whether memory is managed automatically or not, the underlying 
system ultimately requires basic allocation and reclamation primitives such as 
\emph{malloc} and \emph{free}.
%
Unfortunately, out-of-the-box \emph{malloc/free} is not suitable for embedded 
systems, which have different requirements in comparison to standard desktop 
systems.
%
For instance, many memory management hazards that are neglected in larger 
systems can, instead, be critical in embedded systems:

\begin{enumerate}
\item \textbf{Memory corruption}
    Many embedded systems lack memory protection, and continuous allocations in 
    the heap may end up corrupting the stack (and vice versa).

\item \textbf{Run-time overhead}
    Memory management requires extra run-time bookkeeping.
    Also, in the context of embedded systems, a predictable execution model can 
    be even more important than the fastest scheme on the average.

\item \textbf{Metadata overhead}
    Metadata used by the memory manager can spend precious bytes (e.g. linked 
    lists of free blocks).

\item \textbf{Memory fragmentation}
    For constrained memory platforms, unusable holes between and inside 
    allocated blocks (external and internal fragmentation, respectively) can 
waste a big percentage of available memory.

\item \textbf{Unreproducible execution}
    Successive executions of the same program may allocate memory in different 
    ways, possibly leading to different outcomes (e.g. an allocation fail).

\item \textbf{Reclamation hazards}
    Properly reclaiming memory is far from trivial.
    A missed reclamation leads to a memory leak that wastes memory, while 
    reclaiming a memory block still in use leads to a dangling pointer that 
will eventually crash the application.
\end{enumerate}

Note that reclamation hazards are inherent to systems that require an explicit 
free operation, because its correct use is left to programmers.
As an alternative, garbage collection eliminates reclamation hazards, but may 
incur unacceptable run-time overheads.

Note also that as the C standard is not specific about memory allocation 
behavior, \emph{malloc/free} can perform bad in all enumerated issues.
%
Hence, higher-level systems built on top of flawed primitives would inherit the 
same weaknesses.

A common alternative for embedded systems is to use \emph{memory pools} as the 
primitive scheme to manage dynamic allocation.
A memory pool is a static buffer of \emph{N} predefined fixed-sized blocks of 
memory that can be used by an application.
%
In the context of sensor networks, both TinyOS and Coniki operating systems 
offer%
\footnote{
TinyOS provides the \emph{Pool} component, while Contiki, the \emph{memb} 
library.
}
and promote the use of memory pools because most threats raised above can be 
alleviated:

\begin{enumerate}
\item
Given that the whole space for memory pools is static, dynamic allocations will 
never corrupt the stack.
For the same reason, the exact maximum stack limit can be known at compile 
time, helping to evaluate the risks of its growth during runtime.

\item
The run-time overhead is minimal because implementations typically use simple 
arrays to hold the memory blocks.
For instance, both allocation and reclamation are \emph{O(1)} in TinyOS.

\item
As associated metadata, TinyOS and Contiki pools use a single auxiliary vector 
of size \emph{N} to hold pointers for free blocks.

\item
Regardless of how allocation patterns in applications vary, memory pools will 
always guarantee a minimal \emph{N} of memory blocks.
Hence, external fragmentation is non-existent.
Internal fragmentation, however, can be an issue and is discussed further.

\item
Given that memory operations are simple and only handle fixed-size blocks, the 
execution is always deterministic and predictable.

\item
Memory pools are still manipulated through malloc/free-like operations.
Hence, all challenges to properly reclaim memory still hold.
\end{enumerate}

Internal fragmentation occurs when an allocated memory block is bigger than the 
requested size.
Given that memory pools can only handle fixed-size blocks, any allocation that 
requests smaller blocks will contain internal fragmentation (while allocation 
of bigger blocks always fail).
For applications that require blocks of different sizes, memory pools
This is the case for most Java applications, in which all non-primitive types 
are dynamically allocated.

In our system, we promote the use of memory pools as the underlying allocation 
mechanism, but also support heap allocation if more flexibility is required.

\section{Dynamic Abstractions in \CEU}

- synchronous semantics
- like Esterel, similar to Protothreads
- differ from proto compositions, timers,events,locals,fins

\subsection{\CEU Basics}

\begin{comment}
- deterministic and bounded execution
- shared memory concurrency
- integration with C
- locals and finalization
- first-class timers
- internal events
\end{comment}

\CEU is a synchronous reactive language that supports multiple lines of 
execution---known as \emph{trails}---that continuously react to external input 
events broadcast by the environment.
%
By following the synchronous model, each trail reacts to (or ignores) an 
occurring event with a \emph{run-to-completion} policy in the order they appear 
in the code: the scheduler of \CEU is deterministic and non-preemptive.
%
Reactions to different events never overlap, and the compiler ensures that 
programs do not contain unbounded loops, which could lead to unresponsiveness.

The example in Figure~\ref{fig.blink} blinks three LEDs in parallel with 
different frequencies.
%
Symbols defined externally in $C$%
\footnote{Symbols for functions, types, globals, and constants available in the 
$C$ compiler for the target embedded platform.
}, such as \code{LED\_toggle} in the example, can be used seamlessly in \CEU by 
prefixing their names with an underscore.
%
Note that every 1s the three trails awake at the same time, but the 
deterministic scheduler follows the order they appear in the code (i.e., first 
\emph{LED 0} toggles, then \emph{LED 1}, and then \emph{LED 2}).

\begin{figure}[h]
\begin{lstlisting}[numbers=left,xleftmargin=2em]
par do
    every 250ms do
        _LED_toggle(0);
    end
with
    every 500ms do
        _LED_toggle(1);
    end
with
    every 1000ms do
        _LED_toggle(2);
    end
end
\end{lstlisting}
\rule{12.2cm}{0.37pt}
\caption{
    The three LEDs blink every 250ms, 500ms, and 1000ms, respectively.
    \label{lst.TODO}
}
\end{figure}

Programs in \CEU are structured with standard imperative primitives, such as 
sequences, loops, and assignments.
The extra support for parallelism allows programs to wait for multiple events 
at the same time.
Trails await events without loosing context information, such as locals and the 
program counter, what is a desired behavior in concurrent 
applications.~\cite{sync_async.cooperative}
%
The conjunction of parallelism with standard control-flow enable hierarchical 
compositions, in which self-contained blocks of code can be deployed 
independently.
%
To illustrate the expressiveness of compositions in \CEU, consider the two 
variations of the structure that follows:

\begin{minipage}[t]{0.35\linewidth}
\begin{lstlisting}
loop do
    par/and do
        ...
    with
        await 1s;
    end
end
\end{lstlisting}
\end{minipage}
%
\hspace{1cm}
%
\begin{minipage}[t]{0.35\linewidth}
\begin{lstlisting}
loop do
    par/or do
        ...
    with
        await 1s;
    end
end
\end{lstlisting}
\end{minipage}

In the \code{par/and} loop variation, the code in the first trail (represented 
as \code{...}) is repeated every second at minimum, as the second trail must 
also terminate to rejoin the \code{par/and} primitive and restart the loop.
%
In the \code{par/or} loop variation, if the code does not terminate within a 
second, the second trail rejoins the composition (cancelling the first trail) 
and restarts the loop.
%
The two structures represent, respectively, the \emph{sampling} and 
\emph{timeout} patterns, which are common in embedded applications.
%
Note that the body of \textbf{\code{...}} may contain arbitrary code with 
nested compositions and awaits, but the described patterns will work as 
expected.

As a more complete example, we propose to a simple packet forwarder from a set 
of primitives depicted in Figure~\ref{fig.fwd.prims}.

\begin{lstlisting}
 == TYPES:
    * PKT_t
        the packet data type for the protocol

 == FUNCTIONS:
    * PKT_send
        transmits a protocol packet
    * PKT_cancel
        cancels a packet transmition
    * PKT_next
        gets the next hop to forward

 == EVENTS:
    * START
        a request to start the protocol
    * STOP
        a request to stop the protocol
    * RECEIVE
        an arriving protocol packet
    * SENDDONE
        a packet transmition acknowledgement
\end{lstlisting}

\begin{figure}[h]
\begin{lstlisting}[numbers=left,xleftmargin=2em]
loop do
    await START;
    par/or do
        await STOP;
    with
        loop do
            var _PKT_t* pkt = await RECEIVE;
            loop do
                _PKT_send(_PKT_next(pkt), pkt)
                    finalize with
                        _PKT_cancel(pkt);
                    end;
                var _PKT_t* sent;
                var int ok;
                (sent,ok) = await SENDDONE;
                if sent==pkt and ok==1 then
                    break;
                end
            end
        end
    end
end
\end{lstlisting}
\rule{12.2cm}{0.37pt}
\caption{ A simple packet forwarder in \CEU.\newline
{\small %\textmd{
TODO.
}%}
\label{lst.TODO}
}
\end{figure}

- assumption, next always succeed

\begin{itemize}
\item sequential reasoning
\item no state variables (as a consequence)
\item locals (as a consequence)
\item parallel compositions
\item finalize safe (and enforced) termination
\end{itemize}

TinyOS
- 4 callbacks

Protothreads
- state vars for start/stop

\subsection{A Simple Packet Forwarder}

Input events
C operations

\subsection{\CEU Organisms}

\begin{figure}[h]
\begin{lstlisting}[numbers=left,xleftmargin=2em]
class Blink with
    var int led;
    var int dt;
do
    every (dt)ms do
        _LED_toggle(led);
    end
end

do
    var Blink led0 with
        this.led = 0;
        this.dt  = 250;
    end;

    var Blink led1 with
        this.led = 1;
        this.dt  = 500;
    end;

    var Blink led2 with
        this.led = 2;
        this.dt  = 1000;
    end;

    await FOREVER;
end
\end{lstlisting}
\rule{12.2cm}{0.37pt}
\caption{ TODO.\newline
{\small %\textmd{
TODO.
}%}
\label{lst.TODO}
}
\end{figure}

\begin{figure}[h]
\begin{lstlisting}[numbers=left,xleftmargin=2em]
class Forwarder with
    var _PKT_t pkt;
do
    _PKT_send(_PKT_next(&pkt), &pkt)
        finalize with
            _PKT_cancel(&pkt);
        end;
    var _PKT_t* sent = await SENDDONE;
    if sent == &pkt then
        break;
    end
end

loop do
    await START;
    par/or do
        await STOP;
    with
        do
            loop do
                var _PKT_t* pkt = await RECEIVE;
                spawn [10] Forwarder with
                    this.pkt = *pkt;
                end;
            end
        end
    end
end
\end{lstlisting}
\rule{12.2cm}{0.37pt}
\caption{ TODO.\newline
{\small %\textmd{
TODO.
}%}
\label{lst.TODO}
}
\end{figure}

\section{Evaluation}

\section{Conclusion}

\bibliographystyle{splncs}
\bibliography{my,other}
\end{document}
